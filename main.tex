\documentclass[24pt, a4paper, oneside]{book} % oneside prevents different inner/outer margins

% ------------------- Packages -------------------
\usepackage[utf8]{inputenc} % UTF-8 input encoding
\usepackage{fontspec} % For Nepali fonts
\setmainfont{Noto Serif Devanagari} % Change Here: Any installed Nepali font
\usepackage{geometry} % Page layout
\usepackage{fancyhdr} % Header & Footer
\usepackage{setspace} % Line spacing
\usepackage{tocloft}  % Table of Contents customization
\usepackage{titlesec} % Section styling
\usepackage{graphicx} % For images
\usepackage{xstring}  % For string substitution
\usepackage{fmtcount} % For devanagari numerals
\usepackage[percent]{overpic} % Overpic with percent option
\usepackage{multicol} % Multiple columns
\usepackage{tikz} % TikZ graphics
\usepackage{ragged2e} % Text alignment options
\usepackage{microtype} % Better typography
\usepackage{etoolbox} % Programming tools
\usepackage{afterpage} % Delayed page commands
\usepackage{lipsum} % Dummy text
\usepackage{xcolor} % Colors
\usetikzlibrary{calc} % TikZ calculations

% ------------------- Page Layout -------------------
\geometry{
  top=1.5cm,
  bottom=2cm,
  left=2cm,
  right=2cm
}
\setstretch{1.7} % Comfortable line spacing

% ------------------- Front Cover -------------------
% ------------- Full A4 Front Cover (No Margin, No Shift) -------------
\newpage
\thispagestyle{empty}
\begin{picture}(0,0)
  \put(-75,-790){\includegraphics[width=\paperwidth,height=\paperheight]{FRONT-PAGE.jpg}}
\end{picture}
\newpage

% ------------------- Devanagari Section Numbering -------------------
\renewcommand{\thesection}{\devanagaridigits{\arabic{section}}}

% ------------------- Table of Contents Depth -------------------
\setcounter{tocdepth}{1} % Only sections in TOC, no subsections

% ------------------- Custom Song Title for Footer -------------------
\newcommand{\songtitle}{}

% ------------------- Header & Footer -------------------
\pagestyle{fancy}
\fancyhf{}
\fancyfoot[L]{\textit{\fontsize{15pt}{15pt}\selectfont\fontspec{Noto Serif Devanagari}\songtitle}}
\fancyfoot[R]{\thepage}
\renewcommand{\headrulewidth}{0pt}
\renewcommand{\footrulewidth}{0.4pt}

% ------------------- Section (Song Title) Formatting with Horizontal Line -------------------
\titlespacing*{\section}{0pt}{0pt}{0.3em}  % adjust last value to control space below section title

\titleformat{\section}[block]
  {\fontsize{28pt}{20pt}\selectfont\bfseries\centering}{}{0em}{}
  [
  \noindent\rule{\textwidth}{2pt}
  ]

% ------------------- Song Content Styling -------------------
\newenvironment{songtextblock}{
  \setstretch{1.5}
  \fontsize{20pt}{20pt}\selectfont
  \bfseries
}{}



\newenvironment{songtextblock}{
  \setstretch{1.5}
  \fontsize{20pt}{20pt}\selectfont
  \bfseries
}{}
% Custom justified line environment
\usepackage{expl3,xparse}

\ExplSyntaxOn

% Create an environment that splits at line breaks and auto-stretches
\NewDocumentEnvironment{justifylines}{+b}
 {
  \begin{songtextblock}
  \seq_set_split:Nnn \l_tmpa_seq { \\ } { #1 }
  \seq_map_inline:Nn \l_tmpa_seq
   {
     \makebox[\linewidth][s]{##1}\\
   }
 }
 {
  \end{songtextblock}
 }
\ExplSyntaxOff




% ---------- Dots in Table of Contents ----------
\renewcommand{\cftsecleader}{\cftdotfill{\cftdotsep}}

% ---------- Make TOC Entries Bold ----------
\renewcommand{\cftsecfont}{\Large\bfseries}        % Section title bold
\renewcommand{\cftsubsecfont}{\bfseries}     % Subsection title bold
\renewcommand{\cftsubsubsecfont}{\bfseries}  % Subsubsection title bold

\renewcommand{\cftsecpagefont}{\bfseries}        % Section page number bold
\renewcommand{\cftsubsecpagefont}{\bfseries}     % Subsection page number bold
\renewcommand{\cftsubsubsecpagefont}{\bfseries}  % Subsubsection page number bold

% ---------- Beautiful TOC Title ----------

\renewcommand{\contentsname}{\Huge\centering\bfseries\underline{विषय सूची}\\[1mm]}

% ---------- Reduce Top Space Before TOC ----------
\makeatletter
\patchcmd{\@starttoc}{\chapter*{\contentsname}}{%
  \vspace*{0.5cm} % Adjust this to control vertical spacing
  \chapter*{\contentsname}
}{}{}
\makeatother



% ------------------- Document Start -------------------
\begin{document}
% ------------------- Cover Page -------------------
\begin{titlepage}
\newpage
\thispagestyle{empty}

% Top "श्री हरि" with vertical lines
\vspace*{0.5cm}
\begin{center}
    \fontsize{20pt}{20pt}\selectfont
    \bfseries
   \hspace{0.1cm}॥ श्री हरि: ॥\hspace{0.1cm}
\end{center}

% Title in large font
\vspace{1cm}
\begin{center}
    \fontsize{42pt}{42pt}\selectfont
    \bfseries
  श्रीमद्भगवत् गीता रागमाला
\end{center}

% Centered Verse with thick lines
\vfill

% Center the top thick rule
\begin{center}
  \rule{0.45\linewidth}{5pt}  % Top thick line, 50% width
\end{center}

\vspace{-0.5cm}

% Center a minipage with fixed width (same as line width)
\begin{center}
  \begin{minipage}{0.47\linewidth}
    \fontsize{24pt}{24pt}\selectfont % 42pt font, 44pt line height
    \bfseries
    \setlength{\parindent}{2pt}
    % Justify lines with \\ for line breaks
    त्वमेव माता च पिता त्वमेव\\
    त्वमेव \hspace*{0.0050cm}बन्धुश्च\hspace*{0.0050cm} सखा \hspace*{0.010cm}त्वमेव ।\\
    त्वमेव  विद्या द्रविणं त्वमेव
    त्वमेव \hspace*{0.30cm} सर्वं \hspace*{0.25cm} मम देवदेव ॥
  \end{minipage}
\end{center}

\vspace{-0.5cm}

% Bottom thick line same width
\begin{center}
  \rule{0.45\linewidth}{5pt}
\end{center}

\vfill

% Bottom Writer Info
\begin{center}
    \fontsize{24pt}{24pt}\selectfont
   सङ्कलनकर्ता~:~\textbf{.................}
\end{center}
\end{titlepage}

% ------------------- Writer Information -------------------
\vspace*{0.5cm}
\begin{center}
    \fontsize{20pt}{20pt}\selectfont
    \bfseries
   \hspace{0.1cm}॥ श्री हरि: ॥\hspace{0.1cm}
\end{center}
\section*{सङ्कलक को तर्फबाट}
\vspace{0cm}
\noindent{\fontsize{22pt}{22pt}\selectfont
\begin{center}
\textbf{
...........................................
}
\end{center}
......................................
\\
......................................
}%
\vspace{0.5cm}


\vspace*{0.5cm}
\begin{center}
    \fontsize{20pt}{20pt}\selectfont
    \bfseries
   \hspace{0.1cm}॥ श्री हरि: ॥\hspace{0.1cm}
\end{center}
\section*{अथ बिम्व बस्ने विधिहरू}
\vspace{0cm}
\noindent{\fontsize{22pt}{22pt}\selectfont

 अन्तिममा नौ रत्न को फेद टुप्पो पढेर गीता महात्म्य पुष्पाञ्जलि गरेर उत्तरांग पंच आह्वान गरेर समाप्त हुन्छ। 
 यस्ता विम्व बस्दा दशमी, द्वादशी त्रयोदशी तिथिमा उठ्ने‌गरि थाल्नु पर्छ । अरु एक राते बिम्व बस्दा खेरी साईत हेर्न पर्दैन । 
 साथै मंगलबारे बिम्व बस्नलाई हरि सयनी एकादशी पाठ फिराएपछि को पहिलो मंगलबार बाट थाल्नु पर्छ यसमा कुनै साइत हेर्न पर्दैन । 
 त्यसपछि मंगलबारे बिम्व बस्नलाई मंगलचौथि बाट बिम्व थाल्नु पर्छ । कसैले अठार मंगलबारको संकल्पगरेर थाल्छन् भने कसैले आठ मंगलबार मात्र बस्ने गरि थाल्छन् । 
 अरु चाडपर्व हरुमा पनि बिम्व बस्नु पर्छ, जस्तै श्री कृष्ण जन्माष्टमी, बाला चतुर्दशी, शिव रात्री, राम नवमी यस्तै यस्तै बेलामा बिम्व बसिन्छ । 
 बिम्व बस्नु भनेको भगवान ईश्वर को चर्चागर्नु सत्संग गर्नु आत्मा परमात्मा को खोजि गर्नु  ज्ञान हितोपदेशका कुरा गर्नु शास्त्रार्थ काकुरा गर्नु शान्ति र आनन्द रहनुनै मूख्य उद्देश्य हो ।
\vspace{0.5cm}
\begin{center}\HUGE
 \textbf{ईतिश्री ॥\\}
\end{center}
\newpage

% ------------------- Table of Contents -------------------
\tableofcontents
\newpage

% ------------- Full A4 Back Cover (No Margin, No Shift) -------------
\newpage
\thispagestyle{empty} % Removes header and footer
\begin{tikzpicture}[remember picture, overlay]
    \node at (current page.center) {
        \includegraphics[
            width=\dimexpr\paperwidth-3cm\relax,
            height=\dimexpr\paperheight-3cm\relax
        ]{SLOKAS/front-images/15.png}
    };
\end{tikzpicture}
\newpage

\thispagestyle{empty} % Removes header and footer
\begin{tikzpicture}[remember picture, overlay]
    \node at (current page.center) {
        \includegraphics[
            width=\dimexpr\paperwidth-3cm\relax,
            height=\dimexpr\paperheight-3cm\relax
        ]{SLOKAS/front-images/16.png}
    };
\end{tikzpicture}
\newpage

\thispagestyle{empty} % Removes header and footer
\begin{tikzpicture}[remember picture, overlay]
    \node at (current page.center) {
        \includegraphics[
            width=\dimexpr\paperwidth-3cm\relax,
            height=\dimexpr\paperheight-3cm\relax
        ]{SLOKAS/front-images/17.png}
    };
\end{tikzpicture}
\newpage
\thispagestyle{empty} % Removes header and footer
\begin{tikzpicture}[remember picture, overlay]
    \node at (current page.center) {
        \includegraphics[
            width=\dimexpr\paperwidth-3cm\relax,
            height=\dimexpr\paperheight-3cm\relax
        ]{SLOKAS/front-images/18.png}
    };
\end{tikzpicture}
\newpage
\thispagestyle{empty} % Removes header and footer
\begin{tikzpicture}[remember picture, overlay]
    \node at (current page.center) {
        \includegraphics[
            width=\dimexpr\paperwidth-3cm\relax,
            height=\dimexpr\paperheight-3cm\relax
        ]{SLOKAS/front-images/19.png}
    };
\end{tikzpicture}
\newpage
% ------------------- Song Pages -------------------
% ------------------- Song Pages -------------------
\renewcommand{\songtitle}{अथ मङ्गलाचरण प्रारम्भ}
\section{अथ मङ्गलाचरण प्रारम्भ}
\begin{songtextblock}
\begin{justifylines}
ॐ गुरु श्रीमन नारायणाय नमः ।\\
ॐ तत्सत श्रिगुरु चरणम शरणम् नमोनम: ।\\
ॐअविरल मदजल निबहं  भ्रमरकुलाने
  कस्यवित कपोलं‌ ।\\ 
  अभिमत फल दाताराम
कामेशं गुरु गणपतिबन्धे ।\\

\end{justifylines}
 
\begin{center}\Large
   इति श्रीदेवस्तुतिः
  मङ्गलाचरणसमाप्तम् ॥\\
\end{center}


\end{songtextblock}
\newpage
% ------------------ Image Insertion Block ------------------
\begin{center}
  \includegraphics[width=0.70\paperwidth]{A3-1.png}
\end{center}
\vspace{-1cm}
% ------------------ Image Insertion Block ------------------
\vspace{1em} % optional spacing above the image
\begin{center}
  \includegraphics[width=0.70\paperwidth]{A1.png}
\end{center}
\vspace{1em} % optional spacing below the image
\newpage

\renewcommand{\songtitle}{अथ गणेश स्तुति}
\section{अथ गणेश स्तुति}

\begin{songtextblock}
हरि गुरु सिद्धिका कारणजोराभक्तिका
  मञ्चोरा ।\\
  विघ्नका बञ्चोरा पार्वतीका छोरा ॥\\
  हातमा लड्डुकेराजस्का छन उमेरा ।\\
  विघ्नको नाशगरून् ढोग्दिया सातफेरा ॥\\


\begin{center}
\begin{minipage}[c]{0.80\textwidth} % Text on the left
\Large
इति श्री सत्य सीताजी रामगिरी गणेश स्तुति रागमाला संक्षेप वरणं ।\\
  पापहरणं मोक्षदाता जनार्दन शुभम्  ॥
\end{minipage}%
\hfill
\begin{minipage}[c]{0.2\textwidth} % Image on the right
\includegraphics[width=\linewidth]{3.png}
\end{minipage}
\end{center}




\end{songtextblock}



\newpage
\renewcommand{\songtitle}{अथ तिसा यन्त्र रागमाला}
\section{अथ तिसा यन्त्र रागमाला}

\begin{songtextblock}
\begin{justifylines}
   हरि गुरु  अहे ग्रामकि त्रास भयारभलै मलयाचल वासी प्रवास लियो ०२॥\\ 
  तन ताप मेटाउन आसचलो तहिनाचल जाईन, होई जियो ॥\\ 
ल खौफुले रसालकि मौलिपमौल होवै मो दित कोकिल कुककियो ।\\
  तीनकि कलकोमल मन्द महा मधुरीधुनि वाणीमै कान दियो ०२॥\\ 
  \\
  हरि गुरु  मेहतिब्याकुल गोकुलकेलखौगोकुल गोपि ग्वाला पियारे ।\\
बाहा उठाई सहायके हेतु उचाई धरो गिरिनन्द दुलारे ।\\
सानन्द गोपसुता सब होई भुजा चमकै अंगराग छिनारे ।\\
सोभुज मुल करौं अनुकूल हरौ‌‌ भव केसबसुलतिहारे ।\\
मेहतिब्याकुल गोकुलकेलखौगोकुल गोपि ग्वाला पियारे ।\\


\end{justifylines}
 
%\end{justifylines}
%this is image section 
%.......................................
%.........................................
\begin{center}
\begin{minipage}[c]{0.80\textwidth} % Text on the left
\Large
  इति श्री सत्य सीताजि रामगिरी पैलानम्बर बाललिला कबित रागमाला संक्षेप वरणं ।\\
       पापहरणं मोक्षेदाता जनार्दन शुभं ! मधुरबाणि धुनि दिर्घ उच्चारणं समाप्त ॥
\end{minipage}%
\hfill
\begin{minipage}[c]{0.2\textwidth} % Image on the right
\includegraphics[width=\linewidth]{3.png}
\end{minipage}
\end{center} 
\end{songtextblock}

\restoregeometry % go back to normal margins


\newpage
\renewcommand{\songtitle}{अथ द्रोण स्तुति भाषा श्लोक }
\section{अथ द्रोण स्तुति भाषा श्लोक}

\begin{songtextblock}
\begin{justifylines}
    
हरि गुरु  निरञ्जनं निराकारं निराधारं निरामय शरणंसार गुरु महाराज ।\\
गुरुं महाराजं हरी गुरुं महाराज ०२॥\\ 
निरञ्जनं निराकारं निराधारं निरामय शरणंसार गुरु महाराज ।\\
गुरुं महाराजं हरी गुरुं महाराज ०२॥ मनमधिराज अधिराज ।\\
हरि गुरु रातदिन श्रीप्रभूका कमल चरणको ध्यान गर्नु मनले पनि ।\\
व्यर्थैकाबकबात छोडी हरिको नाम भज्नु ठुलो भनि ॥\\ 
नामले तर्छ जगत्, सवैत दुनिया यज्ञादिकोहि जहा ।\\
\\

\end{justifylines}

\begin{center}
    
\begin{tikzpicture}
    % Image node
    \node[inner sep=0pt] (img) at (0,0) {\includegraphics[width=0.7\paperwidth]{f.png}};
    
    % Overlay text node, centered and moved 1cm up
    \node at ($(img.center) + (0,3.5cm)$) [text width=0.7\paperwidth, align=center] {\Large\color{black}
       इति श्री सत्य सीताजि रामगिरी श्रीमन् महाभारते द्रोण पर्व, द्रोण स्तुति रागमाला संक्षेप वरणं ।\\
       पापहरणं मोक्षेदाता जनार्दन शुभं ! मधुरबाणि धुनि दिर्घ उच्चारणं समाप्त ॥
    };
\end{tikzpicture}

\end{center}

\end{songtextblock}


\newpage
\section*{श्रीमद्भगवत गीता को रहस्य}

\subsection*{श्लोक संख्या}
\fontsize{20pt}{18pt}\selectfont
% your content here
% your content here






\begin{itemize}
    \item श्रीमद्भगवत गीतामा जम्मा \textbf{७०४ श्लोक} छन्।
    \item बजारमा आएका सबै पुस्तकमा \textbf{७०० श्लोक} मात्रै हुन्छन्।
    \item तर गीता गुरुमुखि वा सम्प्रदायिक गीता पाठमा \textbf{७०४ श्लोक} पाइन्छन्।
\end{itemize}

\vspace{0.5em}
\subsection*{अतिरिक्त चार श्लोक}


\begin{verse}
\textbf{१. अध्याय ३, श्लोक ४३:} \\
स्वधर्मेण यमाराध्य भक्तामुक्तिमिताबुधाः।\\
त्वं कृष्ण परमानन्दं तोषयत्सर्वकर्मभिः॥ \\

\textbf{२. अध्याय १०, श्लोक ३९:} \\
औषधीनां यवश्चास्मि धातूनामस्मि काञ्चनम्।\\
सर्वेषां स्त्रीणां जातीनां दारुओऽहं पाण्डुनन्दन॥ \\

\textbf{३. अध्याय ११, श्लोक ४५:} \\
ईमानि कर्माणि तवाद्भुतानि कृतानि पूर्वे मुनयो वदन्ति।\\
न ते गुणानां परिमा नमस्ते न तेजसा चास्मि बलस्य विष्णो॥ \\

\textbf{४. अध्याय १३, श्लोक १:} \\
प्रकृतिं पुरुषं चैव क्षेत्रं क्षेत्रज्ञमेव च।\\
एतद्वेदितुमिच्छामि ज्ञानं ज्ञेयं च केशव॥
\end{verse}

\vspace{0.5em}
{\fontsize{16pt}{20pt}\selectfont\bfseries श्लोक रचना जानकारी\par}
\vspace{0.5em}
\begin{itemize}
    \item श्लोकहरू \textbf{अनुष्टुप् छन्द} मा छन्। प्रत्येक श्लोकमा आठ–आठ अक्षरको विश्राम।
    \item कुल अक्षर संख्या: \textbf{२३,१९९}।
    \item \textbf{‘झ’} को अक्षर जम्मा \textbf{१ पटक} — अध्याय १०, श्लोक ३१ मा।
    \item \textbf{‘स’} नपरेका श्लोकहरू: \textbf{५ वटा}।
\end{itemize}

\vspace{0.5em}
\subsection*{‘स’ नभएका ५ श्लोक}

\begin{verse}
\textbf{१. अध्याय २, श्लोक २८:} \\
अव्यक्तादीनि भूतानि व्यक्तमध्यानि भारत।\\
अव्यक्तनिधनान्येव तत्र का परिदेवना॥ \\

\textbf{२. अध्याय ४, श्लोक २९:} \\
अपाने जुह्वति प्राणं प्राणे पानं तथापरे।\\
प्राणापानगति रुद्ध्वा प्राणायामपरायणाः॥ \\

\textbf{३. अध्याय ७, श्लोक २४:} \\
अव्यक्तं व्यक्तिमापन्नं मन्यन्ते मामबुद्धयः।\\
परं भावमजानन्तो ममाव्ययमनुत्तमम्॥ \\

\textbf{४. अध्याय ८, श्लोक १६:} \\
आ ब्रह्मभुवनालोकाः पुनरावर्तिनोऽर्जुन।\\
मामुपेत्य तु कौन्तेय पुनर्जन्म न विद्यते॥ \\

\textbf{५. अध्याय ९, श्लोक २५:} \\
यान्ति देवव्रता देवान्पितॄन्यान्ति पितृव्रताः।\\
भूतानि यान्ति भूतेज्या यान्ति मद्याजिनोऽपि माम्॥
\end{verse}

\vspace{0.5em}
\subsection*{बाचा विवरण}
\begin{itemize}
    \item \textbf{भगवानु वाच}: ५७७ श्लोक
    \item \textbf{अर्जुन उवाच}: ८६ श्लोक
    \item \textbf{सञ्जय उवाच}: ४० श्लोक
    \item \textbf{धृतराष्ट्र उवाच}: १ श्लोक
    \item \textbf{जम्मा}: ७०४ श्लोक
\end{itemize}

\vspace{1em}
\subsection*{श्रीकृष्णका पुर्खाहरूको नाम}

\begin{multicols}{3}
\begin{enumerate}
\item ययाति
\item सान्तसेन
\item सन्तसेन
\end{enumerate}
\end{multicols}
}

% ------------- Full A4 Back Cover (No Margin, No Shift) -------------
\newpage
\thispagestyle{empty} % Removes header and footer
\begin{tikzpicture}[remember picture, overlay]
    \node at (current page.center) {
        \includegraphics[width=\paperwidth,height=\paperheight]{BACK-WITH-AMA.png}
    };
\end{tikzpicture}
\end{document}
